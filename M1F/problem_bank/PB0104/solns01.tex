\documentclass[10pt]{article}
\usepackage{amsfonts}
\usepackage{a4wide}
\usepackage{amsmath}
\thispagestyle{empty}
% for fancy 2 column lists with letters
\usepackage{multicol}
\usepackage[shortlabels]{enumitem}
\newcommand{\R}{\mathbf{R}}
\newcommand{\Z}{\mathbf{Z}}
\begin{document}


\begin{flushright} KMB,\ 3/10/17\end{flushright}

\noindent{\large \bf M1F Foundations of Analysis, Problem Sheet 1 solutions.}

\medskip
\noindent{\bf 1.}

(a) F ($x=2$ is also a root)

(b) T (it doesn't matter that $x=2$ is a root here)

(c) F ($x=2$ is a problem again)

(d) T (the two roots are $x=1$ and $x=2$ -- can you prove there are no others?)

(e) T ($x=3$ isn't a root but this doesn't matter)

(f) F ($x=3$ isn't a root and this time it matters).

The key thing to understand here is that $P \Rightarrow Q$ means, and \emph{only} means, that if $P$ is true, then $Q$ is true. So, for example, part (e) is true, even though in practice it's a bit weird and unhelpful; the point is that logically it's a true statement.

\medskip{\bf 2.} It is true that $R$ implies~$P$. Here's why. Let's assume~$R$ is true (with the goal of trying to deduce that~$P$ is true). Can~$Q$ be false? No! For if~$Q$ is false then we know from the question that~$R$ will also be false, but we're assuming~$R$ is true. So~$Q$ must be true as well. And then again from the question, $P$ must be true. So if~$R$ is true then~$P$ is true too.

\medskip\noindent{\bf 3.} $P$ is true and $Q$ is false, so $(P\Rightarrow Q)$ is false. Similarly, $R$ is false and $S$ is true, so $(R\Rightarrow S)$ is true. So the question asks whether $(\mbox{false})\Leftarrow(\mbox{true})$, and this is false. So the answer to the question is ``false''.

\medskip\noindent{\bf 4.} Let's consider the possibilities for~$P$.

(i) Say~$P$ is true. Then (a) tells us that either~$Q$ or~$R$ (or both) are true. However (c) tells us that if~$Q$ \emph{and}~$R$ are true, then $\neg P$ is true, which would mean that~$P$ is false, a contradiction. We conclude that if~$P$ is true then \emph{exactly one} of~$Q$ and~$R$ are true.

We've not thought about (b) yet so let's do that now. The right hand side of (b) is $R\vee\neg P$ which (because we're assuming~$R$ is true) is the same as $R\vee(\mbox{false})$, which is the same as~$R$. So (b) just says $\neg Q\Rightarrow R$. We know that exactly one of~$Q$ and~$R$ is true, and the other is false. So (b) either says $(\mbox{false})\Rightarrow(\mbox{false})$ or $(\mbox{true})\Rightarrow(\mbox{true})$, and both of these are indeed true.

Conclusion: if $P$ is true, then either $Q$ is true and $R$ is false, or $Q$ is false and $R$ is true.

(ii) Now let's say~$P$ is false. Then (a) is automatically true whatever~$Q$ and~$R$ are (because false implies anything), and (b) and (c) are also automatically true (because $R\vee\neg P$ and $\neg P$ are both true, and anything implies true).

Conclusion: If~$P$ is false, $Q$ and~$R$ can be arbitrary.

Overall conclusion: we can make no individual deductions about any of~$P$,~$Q$ or~$R$. The complete list of possibilities for $PQR$ is $TTF,TFT,FTT,FTF,FFT,FFF$.

\medskip\noindent {\bf 5.} TFTTTFFT.

The only tricky thing here is to understand that $\{1,2,1\}=\{1,2\}$.

\medskip
\noindent{\bf 6.} TFTTTFTT.

Note: $\{1,2\}\in A$ and $A\in B$, but $\{1,2\}\not\in B$.

\medskip\noindent{\bf 7.}

(a) is false, because $B\subseteq\Z$ by definition, so $\frac{1}{2}\not\in B$, so $\frac{1}{2}\not\in A\cap B$.

(b) is true, because $\frac12\in\R$ and $\left(\frac12\right)^2=\frac14<3$, so $\frac12\in A$ and hence $\frac12\in A\cup B$.

(c) is false, because $x=\frac32\in\R$ satisfies $x^2=9/4<3$ and $x^3=27/8>3$, so $x\in A$ but $x\not\in C$, which means $A\not\subseteq C$.

(d) is true. In fact if $x\in\Z$ and $|x|\geq2$ then $x^2\geq4$, meaning $x\not\in B$. On the other hand $\pm1$ and~$0\in B$, meaning $B=\{-1,0,1\}$. All of these elements are easily checked to be in~$C$.

(e) is not true, because $-100\in C$ (as its cube is less than zero) but $(-100)^2=10^4>3$ so $-100\not\in A$ and $-100\not\in B$.

(f) is not true. To check that these sets are not equal, all we need to do is to find a real number which is in one but not the other. Again I claim $x=-100$ will work. For we've just seen it's in~$C$, so it's definitely in $(A\cap B)\cup C$. However we've also just seen it's not in $A\cup B$, so it's also not in $(A\cup B)\cap C$.

\end{document}
