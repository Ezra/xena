\documentclass[10pt]{article}
\usepackage{amsfonts}
\usepackage{a4wide}
\usepackage{amsmath}
\thispagestyle{empty}
% for fancy 2 column lists with letters
\usepackage{multicol}
\usepackage[shortlabels]{enumitem}
\newcommand{\R}{\mathbf{R}}
\newcommand{\Z}{\mathbf{Z}}
\begin{document}


\begin{flushright} KMB,\ 3/10/17\end{flushright}

\noindent{\large \bf M1F Foundations of Analysis, Problem Sheet 1}

\medskip
\noindent{\bf 1.} Which of the following statements involving an integer~$x$ are true and which are false? Just write T or F, and perhaps also one remark about why you think this is the answer.
\begin{multicols}{2}
\begin{enumerate}[(a)]
\item $x^2-3x+2=0\Rightarrow x=1$.
\item $x^2-3x+2=0\Leftarrow x=1$.
\item $x^2-3x+2=0\iff x=1$.
\item $x^2-3x+2=0\iff x=1\mbox{ or }x=2.$
\item $x^2-3x+2=0\Rightarrow x=1\mbox{ or }x=2\mbox{ or }x=3.$
\item $x^2-3x+2=0\Leftarrow x=1\mbox{ or }x=2\mbox{ or }x=3.$
\end{enumerate}
\end{multicols}
\medskip
\noindent{\bf 2.} Suppose $P$, $Q$ and $R$ are mathematical statements (so they are either true or false). Let's say we know that if $Q$ is true then $P$ is true, and that if $Q$ is false then $R$ is false. Does $R$ imply $P$? Write down either a proof, or a counterexample.

\medskip\noindent{\bf 3.} Say $P$ is true, $Q$ is false, $R$ is false and $S$ is true. Is $(P\Rightarrow Q)\Leftarrow(R\Rightarrow S)$ true or false?

\medskip\noindent{\bf 4.} Say~$P$, $Q$ and~$R$ are true/false mathematical statements, and we know the following:
\begin{enumerate}[(a)]
\item $P\Rightarrow(Q\vee R)$,
\item $\neg Q\Rightarrow (R\vee\neg P)$
\item $(Q\wedge R)\Rightarrow \neg P$.
\end{enumerate}
Can we deduce anything about~$P$, $Q$ or~$R$? For example, is~$R$ definitely false? Write down a complete list of possibilities for the truth values of~$P$, $Q$ and~$R$.

\medskip\noindent {\bf 5${}^*$.} Let $A$ be the set $\{1,2,3,4,5\}$. Which of the following statements are true and which are false? (just write T or F).
\begin{multicols}{2}
\begin{enumerate}[(a)]
\item $1\in A$. 
\item $\{1\}\in A$.
\item $\{1\}\subseteq A$.
\item $\{1,2\}\subseteq A$.
\item $\{1,2,1\}\subseteq A.$
\item $\{1,1\}\in A.$
\item $A\in A$.
\item $A\supseteq A$.
\end{enumerate}
\end{multicols}
\medskip
\noindent{\bf 6.} Now let $A$ be the slightly weirder set $\{1,2,\{1,2\}\}$ and let~$B$ be the even weirder set $\{1,2,A\}$. Which of the following statements are true and which are false? (again just write T or F).
\begin{multicols}{2}
\begin{enumerate}[(a)]
\item $1\in A$. 
\item $\{1\}\in A$.
\item $\{1,2\}\in A$.
\item $\{1,2\}\subseteq A$.
\item $1\in B$.
\item $\{1\}\in B$.
\item $(\{1,2\}\in B)\Rightarrow(1\in A)$.
\item $(\{1,2\}\subseteq B)\vee(1\not\in A).$
\end{enumerate}
\end{multicols}
\medskip\noindent{\bf 7.} Set $A=\{x\in\R\,:\,x^2<3\}$, $B=\{x\in\Z\,:\,x^2<3\}$ and $C=\{x\in\R\,:\,x^3<3\}$. For each statement below, either prove it or disprove it! Be careful with your logic and your exposition.
\begin{multicols}{2}
\begin{enumerate}[(a)]
\item $\frac{1}{2}\in A\cap B.$
\item $\frac{1}{2}\in A\cup B.$
\item $A\subseteq C.$
\item $B\subseteq C.$
\item $C\subseteq A\cup B.$
\item $(A\cap B)\cup C=(A\cup B)\cap C$
\end{enumerate}
\end{multicols}
\end{document}
