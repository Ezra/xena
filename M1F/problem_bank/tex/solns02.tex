\documentclass[10pt]{article}
\usepackage{amsfonts}
\usepackage{a4wide}
\usepackage{amsmath}
\thispagestyle{empty}
% for fancy 2 column lists with letters
\usepackage{multicol}
\usepackage[shortlabels]{enumitem}
\newcommand{\R}{\mathbf{R}}
\newcommand{\Z}{\mathbf{Z}}
\begin{document}


\begin{flushright} KMB,\ 16/10/17\end{flushright}

\noindent{\large \bf M1F Foundations of Analysis, Problem Sheet 2, solutions.}

\medskip
\noindent{\bf 1.} Recall that to check that two sets $A$ and $B$ are equal, one has to do two things: first prove $A\subseteq B$ and then prove $B\subseteq A$.

(a) $\bigcup_{n=0}^\infty[n,n+1)$ equals $[0,\infty)$. Why is this? Because $n\geq0$ in the union, we have $0\leq n<n+1<\infty$, so certainly the union is contained within $[0,\infty)$. Conversely if $r\in(0,\infty)$ then there is some integer $n$ such that $n\leq r<n+1$ (we'll prove this in the course; alternatively you might want to argue that it's ``obvious'' and whether it is or not depends on your viewpoint of what mathematics is). This integer $n$ must be at least zero, as if $n<0$ then $n\leq -1$, so $n+1\leq 0$, which implies $r<n+1\leq 0$, a contradiction. Hence $r\in[n,n+1)$ and $n\geq0$, so this is in the union on the left hand side.

(b) This union is $(0,1]$. For if $n\geq1$ then $1/n>0$ and hence $[1/n,1]\subseteq (0,1]$, so we can deduce that the union is contained within $(0,1]$. Conversely, if $r>0$ then we showed in lectures that there's some positive integer $n$ with $0<1/n<r$ (or maybe this is ``obvious''), and hence $r\in[1/n,1]$.

(c) This union is all of $\R$. It's clearly contained in $\R$, and conversely if $r$ is any real number and we choose an integer $N>0$ with $N>r$, and an integer $M>0$ with $M>-r$, and let $n$ be the maximum of $N$ and $M$, we have $r<N\leq n$ and $-r<M\leq n$ so $-n<r$, and we conclude $r\in(-n,n)$.

(d) The intersection is just $(-1,1)$. For if $n\geq1$ then $-n\leq -1<1\leq n$ and hence $(-1,1)\subseteq(-n,n)$, meaning that $(-1,1)$ is contained in the intersection; conversely the intersection is contained in each of the sets in the intersection and in particular within $(-1,1)$.

\medskip

2) Informally, we are going to argue that there can be no largest element, because if $s$ is in $(0,1)$ then the average of $s$ and 1 will be a bit larger. Let me write this down more formally though.

We prove the result by contradiction. Let's assume for a contradiction that $s$ is a largest element of $(0,1)$. Then let's consider $t:=\frac{s+1}{2}$. Because $s\in(0,1)$ we have $s<1$, and hence $s+1<2$ so $t=\frac{s+1}{2}<1$. Because $s>0$ we have $s+1>0$ and hence $t=\frac{s+1}{2}>0$. We deduce that $t\in(0,1)$. Now if $s$ were a largest element of $(0,1)$ then we must have $t\leq s$, but I claim that in fact $t>s$. For $t-s=\frac{s+1}{2}-\frac{2s}{2}=\frac{1-s}{2}>0$ because $s<1$ hence $1-s>0$.

\medskip

3) By contradiction. Let's say 3 divides $n^2$ but it doesn't divide~$n$. Then the remainder when we divide~$n$ by~3 must be 1 or 2, in other words $n=3m+1$ or $3m+2$. 

In the first case $n^2=(3m+1)^2=9m^2+6m+1=3(3m^2+2m)+1$ is not a multiple of~3.

In the second case $n^2=(3m+2)^2=9m^2+12m+4=3(3m^2+4m+1)+1$ is also not a multiple of~3. 

So in either case we have our contradiction, meaning that if $3$ divides $n^2$ then $3$ must divide~$n$.

\medskip

4) (a) This is false. For example if $a=\sqrt{2}$ and $b=-\sqrt{2}$ then both $a$ and $b$ are irrational, but their sum is zero, which is rational.

(b) This is also false. For example if $a=\sqrt{2}$ and $b=0$ then $ab=0$ is rational.

\medskip

5) (a) This is true. We need to show that if $x\in\R$ is arbitrary, then there exists some $y\in\R$ such that $x+y=2$, and this is easy: we can just let $y=2-x$.

(b) This is not true. The claim is that there is some magical number $y\in\R$ which has the property that whatever real number~$x$ we choose, we will have $x+y=2$. But this cannot be true. Let's prove it by contradiction. Let's assume for a contradiction that such a number~$y$ really did exist, and now let's try some values of~$x$. For example let's choose $x=0$; then we have $y+0=2$ and hence $y=2$. But now let's choose $x=1$; then we must have $2+1=y+1=2$, and hence $3=2$, a contradiction. So no such~$y$ can exist.

\medskip

(6) Note that $\sqrt{2}$, $\sqrt{6}$ and $\sqrt{15}$ are all positive. Let's prove $\sqrt{2}+\sqrt{6}<\sqrt{15}$ by contradiction. So let's assume
$$\sqrt{2}+\sqrt{6}\geq\sqrt{15}$$
(NB lose a mark for $>$; the opposite of $<$ is $\geq$). 

Both sides are positive so we can square both sides and deduce
$$(\sqrt{2}+\sqrt{6})\geq15.$$
Now expand out the bracket and tidy up, to get
$$2\sqrt{12}\geq15-8=7.$$
Again both sides are positive so we can square both sides and conclude
$$48\geq49$$
and this is a contradiction. 

Hence $\sqrt{2}+\sqrt{6}<\sqrt{15}$.

{\bf IMPORTANT NOTE.} If you wrote something like this:
\begin{align*}
&\phantom{\Rightarrow}\sqrt{2}+\sqrt{6}<\sqrt{15}\\
&\Rightarrow2+6+2\sqrt{12}<15\\
&\Rightarrow2\sqrt{12}<7\\
&\Rightarrow48<49
\end{align*}
then you get no marks at all. This is because if $P$ is the statement that $\sqrt{2}+\sqrt{6}<\sqrt{15}$ then the argument just above shows that $P$ implies $48<49$, so $P$ implies something true. What can we deduce about $P$ from this? Nothing! Because true implies true, and false implies true.

If however you wrote
\begin{align*}
&\phantom{\Leftarrow}\sqrt{2}+\sqrt{6}<\sqrt{15}\\
&\Leftarrow2+6+2\sqrt{12}<15\\
&\Leftarrow2\sqrt{12}<7\\
&\Leftarrow48<49
\end{align*}
then this would logically be fine, although arguably it would also be upside-down, and also strictly speaking it doesn't follow from what we proved in the course about inequalities because we only proved $0<a<b$ implies $0<a^2<b^2$ rather than the other way around. Can you see how to prove that $0<a^2<b^2$ and $a,b>0$ implies $a<b$?

\medskip

7) (a) Proof by contradiction. If $\sqrt{2}+\sqrt{3/2}$ were rational, then its square would be too. But its square is $2+3/2+2\sqrt{3}$, and if this were rational then $2\sqrt{3}$ and hence $\sqrt{3}$ would be too, contradicting Q3.

(b) This must be irrational because if it were rational then adding $-1$ would leave it rational, but adding 1 gives part (a) which is irrational.

(c) This is rational because it's zero :P

\end{document}
