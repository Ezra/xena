\documentclass[10pt]{article}
\usepackage{amsfonts}
\usepackage{a4wide}
\usepackage{amsmath}
\usepackage{amssymb}
\usepackage{hyperref}
\thispagestyle{empty}
% for fancy 2 column lists with letters
\usepackage{multicol}
\usepackage[shortlabels]{enumitem}
\newcommand{\R}{\mathbf{R}}
\DeclareMathOperator{\cl}{cl}
\newcommand{\Q}{\mathbf{Q}}
\newcommand{\Z}{\mathbf{Z}}
\input xy \xyoption{all}
\begin{document}
\begin{flushright} KMB,\ 14/12/17\end{flushright}

\noindent{\large \bf M1F Foundations of Analysis, Problem Sheet 10.}

\medskip\noindent{\bf 1.} (One-sided inverses.)

(i) Say $f:X\to Y$ is a function and there exists a function $g:Y\to X$ such that $f\circ g$ is the identity function $Y\to Y$. Prove that~$f$ is surjective. 

(ii) Say $f:X\to Y$ is a function and there exists a function $g:Y\to X$ such that $g\circ f$ is the identity function $X\to X$. Prove that~$f$ is injective.

\medskip\noindent{\bf 2.} This relatively straightforward (if you've understood the idea correctly) question makes sure that you understand the composition of functions, and how it differs from other things like multiplication of functions.

Say $f$ and $g$ are functions from $\R$ to $\R$ defined by $f(x)=x^2+3$ and $g(x)=2x$. Write down explicit formulae for the following functions:

(i) $f\circ g$

(ii) $g\circ f$

(iii) $x\mapsto f(x)g(x)$

(iv) $x\mapsto f(x)+g(x)$

(v) $x\mapsto f(g(x))$.

\medskip\noindent{\bf 3.} Prove that $\circ$ is associative. In other words, prove that if $h:A\to B$ and $g:B\to C$ and $f:C\to D$ then $(f\circ g)\circ h=f\circ(g\circ h)$ (NB these are both functions $A\to D$). This is a great example of a question that is dead easy once you actually figure out what it's asking.

\medskip\noindent{\bf 4.} Say $S$ is a set of size $n$, $T$ is a set of size $m$, and $f:S\to T$ is a function.

(i) Prove that if $f$ is injective then $n\leq m$.

(ii) Prove that if $f$ is surjective then $n\geq m$.

(iii) Prove that if $f$ is bijective then $n=m$.

\medskip\noindent{\bf 5.} 
(i) Prove that if $X$ is a countably infinite set, and $Y\subseteq X$ is an infinite subset, then $Y$ is countably infinite.

(ii) Prove that if $A$ and $B$ are countably infinite sets, then so is $A\cup B$.

(iii) Prove that if $A$ and $B$ are countably infinite sets, then so is $A\times B$ (hint: mimic the proof of countability of $\Q$).

(iv) Are there countably many irrational numbers? Countably many complex numbers? What about $\Q(i)=\{a+bi\,:\,a,b\in\Q\}$?

\medskip\noindent{\bf 6.} (i) Prove that if $1\leq r\leq n$ then $\binom{n+1}{r}=\binom{n}{r}+\binom{n}{r-1}$.

(ii) Prove the binomial theorem $(x+y)^n=\sum_i \binom{n}{i}x^{n-i}y^i$ by induction on~$n$.

\medskip\noindent{\bf 7.} (i) Prove that if $p$ is prime and $0<i<p$ is an integer then $p\nmid i!(p-i)!$. Deduce that $p\mid \binom{p}{i}$.

(ii) Give another proof of Fermat's Little theorem, in the form $\forall a\in\Z_{\geq0}$, $a^p\equiv a$~mod~$p$, this time by induction on~$a$. 

\medskip\noindent{\bf 8.} How many anagrams of ``EMMAMCCOY'' are there? (NB you should include any permutation of the letters, including those that don't form words. Feel free to use a calculator).

\medskip\noindent{\bf 9.} (i) Find the coefficient of $x^{19}$ in $(1+x)^{21}$.

(ii) Find the coefficient of $x$ in $(x^3+1/x)^7$.

(iii) Find the coefficient of $x^6$ in $(1+3x+x^2)^5$.
\end{document}