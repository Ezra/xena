\documentclass[10pt]{article}
\usepackage{amsfonts}
\usepackage{a4wide}
\usepackage{amsmath}
\thispagestyle{empty}
% for fancy 2 column lists with letters
\usepackage{multicol}
\usepackage[shortlabels]{enumitem}
\newcommand{\R}{\mathbf{R}}
\newcommand{\Q}{\mathbf{Q}}
\newcommand{\Z}{\mathbf{Z}}
\begin{document}


\begin{flushright} KMB,\ 30/11/17\end{flushright}

\noindent{\large \bf M1F Foundations of Analysis, Problem Sheet 8.}

\medskip\noindent{\bf 1.} Let~$a$ and~$b$ be coprime positive integers (recall that \emph{coprime} here means $\gcd(a,b)=1$). I open a fast food restaurant which sells chicken nuggets in two sizes -- you can either buy a box with~$a$ nuggets in, or a box with~$b$ nuggets in. Prove that there is some integer~$N$ with the property that for all integers~$m\geq N$, it is possible to buy exactly~$m$ nuggets.

\medskip\noindent{\bf 2${}^*$.} True or false?

(i) If~$a$ and~$b$ are positive integers, and there exist integers~$\lambda$ and~$\mu$ such that $\lambda a+\mu b=1$, then $\gcd(a,b)=1$.

(ii) If~$a$ and~$b$ are positive integers, and there exist integers~$\lambda$ and~$\mu$ such that $\lambda a+\mu b=7$, then $\gcd(a,b)=7$.

\medskip\noindent{\bf 3.} (i) Say~$a$ and~$b$ are coprime positive integers, and $N$ is any integer which is a multiple of~$a$ and of~$b$. Prove that~$N$ is a multiple of~$ab$. Hint: we know that $\lambda a+\mu b=1$ for some $\lambda,\mu\in\Z$; now write $N=N\times(\lambda a+\mu b)$.

(ii) By applying~(i) twice, deduce that if~$p$, $q$ and~$r$ are three distinct primes, then two integers $x$ and~$y$ are congruent modulo~$pqr$ if and only if they are congruent mod~$p$, mod~$q$ and mod~$r$.

(iii) (tough) Consider the set of positive integers $\{2^7-2,3^7-3,4^7-4,\ldots,1000^7-1000\}$. What is the greatest common divisor of all the elements of this set? Feel free to use a calculator to get the hang of this; feel free to use Fermat's Little Theorem and the previous part to nail it.

(iv) (tougher) $561=3\times 11\times 17$. Prove that if $n\in\Z$ then $n^{561}\equiv n$ mod~561. Hence the converse to Fermat's Little Theorem is false.

\medskip\noindent{\bf 4.} For each of the following binary relations on a set~$S$, figure out whether or not the relation is reflexive. Then figure out whether or not it is symmetric. Finally figure out whether or not the relation is transitive.

(i) $S=\R$, $a\sim b$ if and only if $a\leq b$.

(ii) $S=\Z$, $a\sim b$ if and only if $a-b$ is the square of an integer.

(iii) $S=\R$, $a\sim b$ if and only if $a=b^2$.

(iv) $S=\Z$, $a\sim b$ if and only if $a+b=0$.

(v) $S=\R$, $a\sim b$ if and only if $a-b$ is an integer.

(vi) $S=\{1,2,3,4\}$, $a\sim b$ if and only if $a=1$ and $b=3$.

(vii) $S$ is the empty set (and $\sim$ is the only possible binary relation on that set, the empty binary relation). 

\medskip\noindent{\bf 5.} Let~$S=\R$ be the real numbers, and let~$G$ be a subset of $\R$. Define a binary relation~$\sim$ on~$S$ by $a\sim b$ if and only if $b-a\in G$.

(i) Say $0\in G$. Prove that~$\sim$ is reflexive.

(ii) Say~$G$ has the property that $g\in G$ implies $-g\in G$. Check that~$\sim$ symmetric.

(iii) Say~$G$ has the property that if $g\in G$ and $h\in G$ then $g+h\in G$. Check that~$\sim$ is transitive.

(iv) If you can be bothered, also check that the converse to all these statements are true as well (i.e. check that if~$\sim$ is reflexive then $0\in G$, if $\sim$ is symmetric then $g\in G$ implies $-g\in G$ etc). 

Remark: Subsets~$G$ of~$\R$ with these three properties in parts (i)--(iii) are called \emph{subgroups} of~$R$, or, more precisely, additive subgroups (the group law being addition). So this question really proves that the binary relation defined in the question is an equivalence relation if and only if $G$ is a subgroup of~$\R$. You'll learn about groups and subgroups next term in M1P2.
\end{document}