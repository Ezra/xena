\documentclass[10pt]{article}
\usepackage{amsfonts}
\usepackage{a4wide}
\usepackage{amsmath}
\thispagestyle{empty}
% for fancy 2 column lists with letters
\usepackage{multicol}
\usepackage[shortlabels]{enumitem}
\newcommand{\R}{\mathbf{R}}
\newcommand{\Z}{\mathbf{Z}}
\begin{document}


\begin{flushright} KMB,\ 8/11/17\end{flushright}

\noindent{\large \bf M1F Foundations of Analysis, Problem Sheet 5 Solutions.}

\medskip\noindent{\bf 1.} 

a) We first check that $F_3=1+1=2$ is even, and hence that $P(1)$ is true.

For the inductive step, we note that if $P(d)$ is true, then $F_{3d-2},F_{3d-1},F_{3d}$ are odd, odd, even, and now continuing the pattern, the next three terms are odd+even=odd, even+odd=odd, and then odd+odd=even, hence the next three are odd, odd, even as well. Slightly more formally, $F_{3(d+1)-2}=F_{3d+1}=F_{3d-1}+F_{3d}$ is odd, $F_{3(d+1)-1}=F_{3d+2}=F_{3d+1}+F_{3d}$ is odd, and $F_{3(d+1)}=F_{3d+2}+F_{3d=1}$ is even. Hence $P(d+1)$ is true, if $P(d)$ is. This means that $P(n)$ is true for all $n\geq1$ by the principle of mathematical induction.

b) $2+0+1+7=10$ leaves remainder~1 when divided by~3, so~2017 has remainder~1 when divided by~3, so $F_{2017}$ is odd.

\medskip\noindent{\bf 2.} Let $P(n)$ be the statement ``$4^n>3^n+2^n$''. Now $P(1)$ is false, but we don't care. And $P(2)$ is true, because it says ``$16>9+4$'' which is true.

Now let's prove $P(n)$ implies $P(n+1)$ for all $n\geq2$; then our slightly modified principle of mathematical induction will tell us that $P(n)$ is true for all $n\geq2$. And this works, because if $P(n)$ is true, then $4^n>3^n+2^n$, and multiplying both sides by 4 (which is positive) we see $4^{n+1}>4.3^n+4.2^n>3.3^n+2.2^n=3^{n+1}+2^{n+1}$, meaning $P(n+1)$ is true.

Note that $P(n)$ is the true-false statement ``$4^n>3^n+2^n$'', it is not the number $4^n$ or the number $3^n+2^n$, so hopefully nobody is writing anything like ``$P(n)>3^n+2^n$'' in their solutions.

\medskip\noindent{\bf 3.} 

a) For $n\in\Z_{\geq1}$, let $f(n)$ be the number $n+(n+1)+(n+2)+(n+3)$, and let $P(n)$ be the statement that $f(n)$ has remainder~2 when divided by~4. Then $P(1)$ is true, because $f(1)=1+2+3+4=10=2\times4+2$. Moreover, $P(n)$ implies $P(n+1)$, because $f(n+1)=(n+1)+(n+2)+(n+3)+(n+4)=[n+(n+1)+(n+2)+(n+3)]+4=f(n)+4$, hence the remainders when you divide $f(n+1)$ and $f(n)$ by 4 are the same. We're done, by induction.

b) Let $P(n)$ for $n\in\Z_{\geq0}$ be the statement that $11^n-3^n$ is a multiple of~8. Then $P(0)$ is true, because 0 is a multiple of~8. And if $d\geq0$ and $P(d)$ is true, then $11^d-3^d$ is a multiple of 8, and so multiplying by~11 we deduce $11^{d+1}-11.3^d$ is also a multiple of~8. But clearly $8.3^d$ is a multiple of~8 too, and the sum of two multiples of~8 is a multiple of~8, hence
\begin{align*}
&\phantom{=}11^{d+1}-11.3^d+8.3^d\\
&=11^{d+1}-3.3^d\\
&=11^{d+1}-3^{d+1}
\end{align*}
is a multiple of~8. We have shown that $P(d)$ implies $P(d+1)$, so $P(n)$ is true for all $n\geq0$ by induction.

c) For the first one, observe $n+(n+1)+(n+2)+(n+3)=4n+6=4(n+1)+2$, which is hence~2 mod~4. For the second observe that for $n=0$ we can check directly, and for $n\geq1$ we have $11^n-3^n=(11-3)(11^{n-1}+3.11^{n-2}+3^2.11^{n-3}+\cdots+3^{n-2}.11+3^{n-1})$ which is clearly a multiple of~$11-3=8$.

\medskip\noindent{\bf 4.} We see $1!=1<3^1=3$, $2!=2<3^2=9$, $3!=6<3^3=27$, $4!=24<3^4=81$, $5!=120>3^5=243$, and $6!=720<3^6=729$, but $7!=5040>3^7=2187$, so the inequality $n!<3^n$ is true for $n\in\{1,2,3,4,5,6\}$ and false for $n=7$.

Let us now prove that it is false for all $n\geq7$ by induction. For $n\geq7$ define $P(n)$ to be the statement that $n!\geq 3^n$. Then $P(7)$ is true, and for $d\geq7$ we have that $P(d)$ implies $P(d+1)$, because if $P(d)$ is true then $d!\geq 3^d$, so $(d+1)!\geq d.d!\geq 3.d!\geq 3.3^d=3^{d+1}$. 

\medskip\noindent{\bf 5${}^*$.} It's impossible to buy~43 chicken nuggets. Let's prove this by contradiction. Let say we've just managed to buy~43 chicken nuggets. We can't just have boxes of 6 or 9, because both 6 and 9 are multiples of~3, and~43 is not. So we must have bought at least one pack of~20, and somehow bought another~23 as well. But again~23 is not a multiple of~3, so we couldn't have just used boxes of~6 or~9 and so we must have bought a second box of~20. These two boxes make~40 so far, so now we need to buy three more, but we clearly cannot buy~3, because the smallest amount we can buy is~6.

However it is possible to buy any number $m\geq44$ of nuggets. This is easy to prove by induction, once you have done by far the hardest bit, which is figuring out exactly what to prove.

Let me start by showing that we can buy any number of nuggets between~44 and~53:

\begin{align*}
44&=20+4.6\\
45&=9+6.6\\
46&=20+20+6\\
47&=20+3.9\\
48&=8.6\\
49&=20+20+9\\
50&=20+5.6\\
51&=9+7.6\\
52&=20+20+6+6\\
53&=20+9+4.6
\end{align*}
Now for $n\in\Z$, $n\geq8$, let $P(n)$ denote the statement ``it is possible to buy any of the following numbers of chicken nuggets: $6n, 6n+1, 6n+2, 6n+3, 6n+4, 6n+5$''. If we can prove $P(n)$ for all $n\geq8$ then we are home and dry, because for any number $m\geq48$ we can write $m=6q+r$ with $q$ the quotient, at least~8, and~$r$ the remainder with $0\leq r\leq 5$, and then $P(q)$ says we can do it.

Now $P(8)$ is true, because we just checked it explicitly, and clearly $P(d)$ implies $P(d+1)$ because if I can buy $6d+r$ ($0\leq r\leq 5$) then I can buy $6(d+1)+r$ by just buying another box of~6 chicken nuggets.

So we can buy any number $m\geq48$, and also anything between 44 and~47, but not~43, so we are done: the answer is~43.

\medskip\noindent{\bf 6.} For $n\geq1$, let $P(n)$ be the statement that if there are $n$ blue-eyed islanders, then each of them will leave the island on the $n$th boat after the visitor's revelation (and nobody will leave beforehand).

Clearly $P(1)$ is true; the only blue-eyed islander will know that they have blue eyes after the visitor's comments.

Moreover $P(d)$ implies $P(d+1)$; if $P(d)$ is true, and there are $d+1$ islanders with blue eyes, then each of these $d+1$ islanders knows that there are either $d$ or $d+1$ islanders with blue eyes, and after $d$ days, when nobody leaves, each of the blue-eyed islanders will realise that the other blue-eyed islanders can't see $d-1$ blue eyes (or else they would have already left), so their own eyes must be blue, and they will leave on day $d+1$.

Hence $P(n)$ is true for all $n$ by induction.

All islanders, with either blue or brown eyes, know this. Hence after 100 days when all the blue-eyed islanders leave, all the remaining islanders know that they must have brown eyes, so they all leave on day 101.

As for the paradox, I resolve it in the hints: everyone knows there are blue-eyed islanders, and everyone knows that everyone knows there are blue-eyed islanders, etc, etc, but this doesn't go on forever. Take the case of two blue-eyed islanders, for example. Then everyone knows that there are blue-eyed islanders, but one blue-eyed islander does not know for sure that the other blue-eyed islander knows this. Now take the case of three: everyone knows that there are blue-eyed islanders, and everyone knows that everyone knows that there are blue-eyed islanders, but a blue-eyed islander cannot rule out the possibility that there are only two blue-eyed islanders, and we just analysed this case -- so in particular, if there are three blue-eyed islanders, then the blue-eyed islanders do not know for sure that everyone knows that everyone knows that there are blue-eyed islanders. This is what changes when the visitor speaks -- after that, everyone knows there is a blue-eyed islander, and everyone knows that everyone knows, and everyone knows that everyone knows that everyone knows, and so on for as many iterations as you like.
\end{document}