\documentclass[10pt]{article}
\usepackage{amsfonts}
\usepackage{a4wide}
\usepackage{amsmath}
\thispagestyle{empty}
% for fancy 2 column lists with letters
\usepackage{multicol}
\usepackage[shortlabels]{enumitem}
\newcommand{\R}{\mathbf{R}}
\newcommand{\Z}{\mathbf{Z}}
\begin{document}


\begin{flushright} KMB,\ 24/10/17\end{flushright}

\noindent{\large \bf M1F Foundations of Analysis, Problem Sheet 3 Solutions.}

\medskip
\noindent{\bf 1.} If $0<x$ then adding $y$ to both sides and using (A1) we deduce that $y<x+y$. We also know that $0<y$, so $0<y<x+y$ and applying (A2) we deduce $0<x+y$.

\medskip
\noindent{\bf 2$\dag$.}

a) Say $c$ is positive and $y$ is negative. Then $0>y$ so setting $x=0$ in the fact we proved in lectures we deduce $c0>cy$. But $c\times 0=0$, so $0>cy$ and hence $cy$ is negative.

b) Say $x<0$ and $y<0$. Then we know $-x>0$ and $-y>0$, so by (A4) we deduce $(-x)(-y)>0$. Hence $(-1)x(-1)y>0$, so $(-1)^2xy>0$. But $(-1)^2=1$ so $1xy>0$ and hence $xy>0$.

c) Let's prove this by contradiction. Suppose for a contradiction that $x\not=0$ and $y\not=0$. Then by (A3) we must have that $x$ is either positive or negative, and $y$ is either positive or negative. So $xy$ is either the product of two positive numbers, the product of two negative numbers, or the product of one positive and one negative number. In all three cases the product cannot be zero! Indeed, all cases are covered by (A4) and (a) and (b) above, and the observation that positive and negative numbers are non-zero. There is our contradiction, which finishes the proof.

d) Let~$y>0$ satisfy $y^2=x$. We need to find all real solutions to $z^2=x$, or equivalently to $z^2=y^2$. Rewrite as $z^2-y^2=0$ and then factor as $(z+y)(z-y)=0$. By part (c) we deduce that either $z+y=0$ or $z-y=0$, leaving the two possibilities $z=-y$ and $z=+y$. Note that these solutions are definitely different, because $+y>0$ and $-y<0$. Finally note that both work, because $y^2=x$ by definition, and $(-y)^2=(-1)^2y^2=1x=x$.

\medskip
\noindent{\bf 3.}

a) Say $x^{3000000000000}=3$ and $y^{2000000000000}=2$. Then $x^{6000000000000}=3^2=9$ and $y^{6000000000000}=2^3=8$. In particular we must have $x>y$, as $x<y$ would imply $9=x^{6000000000000}<y^{6000000000000}=8$ and $x=y$ would imply $9=8$.

b) $10000^{100}=\left(100^2\right)^{100}=100^{200}$, which is much less than $100^{10000}$.

c) The square root of $2^{22}$ is $2^{11}$ (check by squaring) ands the square root of $2^{2^{22}}$ is $2^{2^{21}}$ (again check by squaring).

\medskip
\noindent{\bf 4${}^*$.} Rewrite as $\frac{3x^2+1}{x}<4$ and then as $\frac{3x^2-4x+1}{x}<0$. Factorise the top to get $\frac{(3x-1)(x-1)}{x}<0$. Note that all of these steps are reversible, so $3x+\frac{1}{x}<4$ \emph{if and only if} $\frac{(3x-1)(x-1)}{x}<0$.

Now the left hand side changes sign at $x=0$, $x-1=0$ and $3x-1=0$, that is, at $x=0$, $x=1$ and $x=1/3$ (and note that $x=0$ is not allowed, and $x=1$, $x=1/3$ don't work because the left hand side is zero). So we consider the four regions $x<0$, $0<x<1/3$, $1/3<x<1$ and $x>1$ separately, and see that $x<0$ and $1/3<x<1$ work, and the other two regions do not. Hence the set of solutions is
$$\{x\in\R\,:\,x<0\mbox{ or }1/3<x<1\}.$$

\medskip
\noindent{\bf 5.} 

a) Say $t>0$ and $|x|<t$. Let's deal with the cases $x\geq0$ and $x<0$ separately.

(i) $x\geq0$. Then $|x|=x$ so the equation becomes $x<t$, and the solutions in this range are $[0,t)$, that is, the reals $x$ with $0\leq x<t$.

(ii) $x<0$. Then $|x|=-x$, so the equation becomes $-x<t$, or equivalently $-t<x$. Hence the solutions in this range are $-t<x<0$, or the open interval $(-t,0)$.

The full set of solutions is hence $(-t,0)\cup[0,t)=(-t,t)$, the reals $x$ with $-t<x<t$.

b) Set $y=x+1$. Then $|y|<3$, or equivalently $-3<y<3$, and so $-3<x+1<3$. Subtracting 1 we see that this is equivalent to $-4<x<2$.

c) This is a little trickier. There are a couple of ways to do it. We could first prove the lemma that $|x|^2=x^2$ for all real numbers~$x$ (by checking the two cases) and then noting that if $a,b\geq0$ then $a<b$ if and only if $a^2<b^2$. Using this line of argument we can deduce that the question is equivalent to asking for all real numbers~$x$ such that $(x-2)^2<(x-4)^2$, and multiplying out and simplifying we see that this equation is equivalent to $-4x+4<-8x+16$ and hence $4x<12$, giving $x<3$ as the answer.

A different approach, more low-level but more tedious, would be to observe that the definition of $|x-2|$ depends on whether $x\geq2$ or $x<2$, and the definition of $|x-4|$ depends on whether $x\geq4$ or $x<4$, so we could deal with the three cases:

i) $x<2$

ii) $2\leq x<4$

iii) $x\geq4$

separately. In case (i) we get $2-x<4-x$ which is always true, so $x<2$ always works. In case (ii) we get $x-2<4-x$, which is true if and only if $2x<6$, that is $x<3$, so $2\leq x<3$ also works. In case (iii) we get $x-2<x-4$, which is never true. So the answer is the union of the sets $\{x\in\R\,:\,x<2\}$ and $\{x\in\R\,:\,2\leq x<3\}$, which is just the real numbers less than~3. We sometimes write this set as $(-\infty,3)$, even though $\infty$ is not a real number.

There is a more geometric approach to this question though. We could regard $|x-2|$ as the distance from $x$ to the real number 2, and $|x-4|$ as the distance from $x$ to the real number~4. Hence the question is asking for the real numbers which are nearer to~2 than to~4, and if you draw the number line you can see that clearly the answer is $x<3$. However in my mind this answer is less ``formal'' than the other two; my instinct would be to use this as a guide and then to use one of the first two approaches. Whether this geometric argument would be appropriate depends on the context, I guess.

\medskip\noindent{\bf 6.} Say $p=a+ib$ and $q=c+id$ and $r=e+if$ are complex numbers.

a) First $(p+q)+r=((a+ib)+(c+id))+(e+if)$ (by definition of $p,q,r$).

This equals $((a+c)+i(b+d))+(e+if)$ (by definition of addition of complex numbers).

This equals $((a+c)+e)+i((b+d)+f)$ (again by definition of addition of complex numbers).

Similarly, $p+(q+r)$ works out as $(a+(c+e))+i(b+(d+f))$.

But we know that $(a+c)+e=a+(c+e)$ and $(b+d)+f=b+(d+f)$, because these are real numbers, so we can assume associativity of addition for these.

Hence $(p+q)+r=((a+c)+e)+i((b+d)+f)=(a+(c+e))+i(b+(d+f))=p+(q+r)$ and we're done.

b) $pq=(a+ib)(c+id)=(ac-bd)+i(ad+bc)$, and $qp=(ca-db)+i(da+cb)$. But $ac=ca$, and $bd=db$, and $ad=da$ and $bc=cb$ because $a,b,c,d$ are real. Hence $(ac-bd)=(ca-db)$ and $(ad+bc)=(da+cb)$, so $pq=qp$.

c) If $p=a+ib$ and $q=c+id$ then $\overline{p}=a-ib$ and $\overline{q}=c-id$, so $\overline{p}\,\overline{q}=(a-ib)(c-id)=(ac-bd)+i(-ad-bc)=(ac-bd)-i(ad+bc)$. On the other hand $pq=(ac-bd)+i(ad+bc)$ so $\overline{pq}=(ac-bd)-i(ad+bc)$ and we are done.

\medskip\noindent{\bf 7.} We know $z^2=-1$ so $(x^2-y^2,2xy)=(-1,0)$ as elements of $\R^2$. This means $x^2-y^2=-1$ and $2xy=0$. Now $2xy=0$ implies $xy=0$ (multiply both sides by $1/2$) and by Q2(c) we deduce that either $x=0$ or $y=0$. We now need to deal with these two cases separately.

(i) $y=0$. Then $2xy=0$ which is great, but $x^2-y^2=-1$ implies $x^2=-1$, which is impossible because we proved in lectures that if $x$ is real then $x^2\geq0$, and $-1<0$. So this can't happen.

(ii) $x=0$. Then $2xy=0$ and $-1=x^2-y^2=0-y^2$, so $y^2=1$. We know $1^2=1$, so by Q2(d) the only possibilities are $y=1$ and $y=-1$, giving $x+iy=i$ or $x+iy=-i$. 

Conversely both $i$ and $-i$ square to $-1$ as is easily seen from the definition, so there are our two answers.
\end{document}