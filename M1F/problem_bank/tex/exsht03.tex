\documentclass[10pt]{article}
\usepackage{amsfonts}
\usepackage{a4wide}
\usepackage{amsmath}
\thispagestyle{empty}
% for fancy 2 column lists with letters
\usepackage{multicol}
\usepackage[shortlabels]{enumitem}
\newcommand{\R}{\mathbf{R}}
\newcommand{\Z}{\mathbf{Z}}
\begin{document}


\begin{flushright} KMB,\ 24/10/17\end{flushright}

\noindent{\large \bf M1F Foundations of Analysis, Problem Sheet 3.}

\medskip
\noindent{\bf 1.} Using only the 4 standard inequality facts (A1) to (A4), write down a proof that if $0<x$ and $0<y$ then $0<x+y$. Hint: you'll only need to use two of them. 

\medskip
\noindent{\bf 2$\dag$.} This question is a little harder and longer than the others, and so I've put a dagger by it. This question continues the development of the basic facts about inequalities for real numbers.

We say that a real number $x$ is \emph{positive} if $x>0$ and \emph{negative} if $x<0$. Recall that (A3) says that every real number is exactly one of: positive, negative, or zero, and (A4) says that the product of two positive numbers is positive. 

a) We proved in lectures that if $x>y$ and $c>0$ then $cx>cy$. Deduce from this that the product of a positive number and a negative number is negative.

b) We showed on the handout (Q2) that if $x<0$ then $-x>0$. For this part you may assume that $-x$ is the product of $-1$ and $x$, that the square of $-1$ is $+1$, and that the product of a bunch of numbers is independent of the order that you multiply them together. Using only these facts, deduce that the product of two negative real numbers is positive.

c) Deduce if $x$ and $y$ are real numbers, and $xy=0$, then either $x=0$ or $y=0$. Hint: rule out all other possibilities using the previous parts.

d) For this part you may assume that if $x>0$ is a real number, then there is a unique positive real number $y>0$ such that $y^2=x$. Prove that there are exactly two real numbers~$z$ such that $z^2=x$, and figure out what they are. Hint: use (c).

\medskip
\noindent{\bf 3.} In this question you may \emph{assume} that if $x>0$ is real and $n>0$ is an integer, then there's a unique positive real $y$, called \emph{the $n$th root of $x$}, such that $y^n=x$. You can also assume all standard results about powers such as $\left(a^b\right)^c=a^{bc}$ and so on.

a) Which is bigger, the three trillionth root of~3 or the two trillionth root of~2? Remark: a trillion is $10^{12}$. Hint: if $0<x<y$ then $0<x^n<y^n$ for $n\geq1$ (you can assume this, or you can prove it by induction if you know about induction).

b) Which is bigger, $100^{10000}$ or $10000^{100}$?

c) What's the square root of $2^{22}$? What's the square root of $2^{2^{22}}$?

\medskip
\noindent{\bf 4${}^*$.} Find the set of non-zero real numbers~$x$ such that $3x+\frac{1}{x}<4$. Hint: \emph{be careful.} It is \emph{not} true that if $a<b$ then $xa<xb$: this is only true for $x>0$.

\medskip
\noindent{\bf 5.} Let's define the \emph{absolute value} $|x|$ of a real number~$x$ by $|x|=x$ if $x\geq0$ and $|x|=-x$ if $x<0$.

a) Prove that if $t$ is a positive real number then $|x|<t$ if and only if $-t<x<t$.

b) Find all real numbers~$x$ such that $|x+1|<3$.

c) Find all real numbers~$x$ such that $|x-2|<|x-4|$.

\medskip\noindent{\bf 6.} Say $p=a+ib$ and $q=c+id$ and $r=e+if$ are complex numbers.

a) Prove from first principles that $(p+q)+r=p+(q+r)$. You may assume that the analogous fact holds for real numbers -- what I'm asking is how to deduce this fact for complex numbers. 

This fact (``it doesn't matter which $+$ you do first when you add three things up''), has a proper fancy name, which is ``associativity of addition''.

b) Prove from first principles that $pq=qp$ (again you may assume that the analogous fact (``it doesn't matter which order you multiply numbers together'') is is true for real numbers, and again it has a fancy name, namely ``commutativity of multiplication'').

c) Recall that the \emph{complex conjugate} of $z=x+iy$ is the complex number $\overline{z}=x-iy$. Prove from first principles that $\overline{p}\,\overline{q}=\overline{pq}$.

\medskip\noindent{\bf 7.} Prove from first principles that if $z=x+iy$ and $z^2=-1$ then $z=i$ or $z=-i$. Hint: Q2(c) and Q2(d) will be helpful. 
\end{document}
