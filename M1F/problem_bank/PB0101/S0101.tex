(PB0101)

(a) F ($x=2$ is also a root)

(b) T (it doesn't matter that $x=2$ is a root here)

(c) F ($x=2$ is a problem again)

(d) T (the two roots are $x=1$ and $x=2$ -- but can you \emph{prove} that there are no others?)

(e) T ($x=3$ isn't a root but this doesn't matter)

(f) F ($x=3$ isn't a root and this time it matters).

The key thing to understand here is that $P \Rightarrow Q$ means, and \emph{only} means, that if $P$ is true, then $Q$ is true. So, for example, part (e) is true, even though in practice it's a bit weird and unhelpful; the point is that logically it's a true statement.
